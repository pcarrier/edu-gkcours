\documentclass{beamer}
\usetheme{Copenhagen}

\usepackage[utf8]{inputenc}
\usepackage[francais]{babel}
\usepackage{hyperref}
\usepackage{textcomp}
\usepackage{xcolor}
\usepackage{graphicx}

\AtBeginSection[]
{
\begin{frame}
\frametitle{Plan}
\tableofcontents[currentsection,hideothersubsections]
\end{frame}
}

\title{Programmation Neuro-linguistique}
\author{Geoffroy Carrier, Jean-Christophe Saad-Dupuy}
\institute{Université Joseph Fourier, L3 MIAGE}

\date{23 février 2009}

\begin{document}

\frame{\titlepage}

\begin{frame}
\frametitle{Plan}
\tableofcontents
\end{frame}

\section{Introduction} % (fold)
\label{sec:introduction}
\begin{frame}[t]\frametitle{Kézako ?}
	Il existe énormément de définitions plus ou moins compatibles...
	\begin{cite}
		
	\end{cite}
\end{frame}

\begin{frame}[t]\frametitle{Objectifs}
	\begin{itemize}
		\item<+-> Modèles de comportements ;
		\item<+-> Techniques pour augmenter les habiletés relationnelles ;
	\end{itemize}
\end{frame}

% section introduction (end)

\section{Présentation historique}
\subsection{Émergence}
\subsection{Croissance}
\frame{
truc
}
\subsection{Récupération}

\section{Théorie}
\subsection{Fondements}
\begin{frame}[t]\frametitle{Experts en communication}
	\begin{itemize}
		\item<+-> Une acuité sensorielle développée ;
		\item<+-> Une capacité à établir le rapport ;
		\item<+-> Un respect réel du modèle du monde de l'autre ;
		\item<+-> Un art de poser des questions précises ;
		\item<+-> Beaucoup de flexibilité et d'adaptabilité ;
		\item<+-> Une adaptation facile ;
		\item<+-> Une aptitude à établir et poursuivre des objectifs spécifiques.
	\end{itemize}
\end{frame}
\subsection{Techniques}

\section{Dans l'environnement professionnel}
\subsection{Un outil pour comprendre}
\subsection{Un outil pour être compris}

\section{Limites et précautions}
\subsection{Une secte ?} % (fold)
\label{sub:une_secte}
\begin{frame}[t]\frametitle{title}
	D'après le rapport 2001 de la Mission Interministérielle de Lutte contre les Sectes~:
	\begin{cite}
	La programmation neurolinguistique, couramment dénommée PNL,
	forme un ensemble disparate de méthodes de communication
	(apprendre à reformuler un message, à décoder des signaux non verbaux,
	des mouvements oculaires, etc..),
	basé sur un ensemble tout aussi disparate de références théoriques.
	Les fondements scientifiques et les validations empiriques sont faibles~:
	les hypothèses relatives aux mouvements oculaires ont d'ailleurs été infirmées. \
	[...] \
	L'absence de principes déontologiques orientés vers l'aide et la santé,
	plutôt que vers l'exploitation et le profit,
	l'absence de connaissances en psychopathologie et en psychiatrie
	permettant d'aider ou d'orienter les personnes perturbées,
	l'absence de formation scientifique permettant de relativiser les connaissances
	et de ne pas prétendre à la vérité, caractérisent les pratiques qui font question.
	\end{cite}
\end{frame}
% subsection une_secte (end)

\end{document}
